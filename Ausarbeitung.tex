\documentclass[onecolumn,german]{article}

\usepackage{Ausarbeitung}

% Bitte hier Ihren Namen eintragen
\author{Sarah Barton und Sonja Vogl \\ Technische Universit�t M�nchen}

\title{IDP im Sommersemester 2013 \\
% Hier bitte den Titel eintragen!
       {\bf QualityControlTool}
}

% Bitte Datum des Vortrags angeben
\date{05. Juli 2013}


\begin{document}
\maketitle

%\begin{figure}
%\centerline{
%\includegraphics[width=0.9\columnwidth]{Abbildungen/TUM-Logo-102.png}
%}
%\label{tum}
%\end{figure}

\begin{abstract}
\noindent
Diese Ausarbeitung soll dem Benutzer einen Einblick in die Bedienbarkeit des QualtiyControlTools geben.
Ausgehend von der ersten Version der beiden Entwickler Simon und bla werden die Erweiterungen des QualityControlTools vorgestellt.
Zudem geht diese Arbeit auf die Funktionen ein, die das QualityControlTool zur Analyse der Biosensoren GeneActive, Somnowatch, Shimmer und GT3X+  bereitstellt.

\end{abstract}

\section{Einleitung}
\label{Einleitung}

Diese Arbeit gibt dem Benutzer des QualityControlTools einen �berblick �ber die M�glichkeiten, die das Programm zur Analyse der Biosensoren GeneActive, Somnowatch, Shimmer und GT3X+ dem User bietet.(siehe Kapitel ...) Zudem wird dem Benutzer nahegebracht, wie das QualityControlTool zu bedienen ist. Hierzu wird zun�chst die Benutzeroberfl�che( siehe Kapitel ...), wie auch die Funktionsweise der einzelnen Buttons ( siehe Kapitel ...) dargestellt. Zus�tzlich wird grob auf die Implementierung eingegangen.

\section{Motivation}

Das Institut f�r Medizinische Statistik und Epidemiologie (IMSE) der TU M�nchen besch�ftigt sich mit der Analyse der Daten der \textsc{Kora} Studien. \textsc{Kora} steht f�r die \textit{kooperative Gesundheitsforschung in der Region Augsburg}.

to do:noch mehr details aus kathrins anfang einf�gen

Das QualityControlTool soll das Auswerten der csv-Dateien von Biosensoren erleichtern. Daf�r existieren zahlreiche Algorithmen, welche in \textsc{Matlab} implementiert sind. Allerdings besitzen viele in der Medizin nur ein eingeschr�nktes Informatikwissen. F�r diese ist eine Auswertung mit einem hohen zeitlichen und hohem Arbeitsaufwand, sich in Matlab einzuarbeiten, verbunden.

Mit dem QualityControlTool ist eine einfache Bedienbarkeit ohne informatische bzw. mathematische Kenntnisse gegeben. Sobald man mit der Benutzeroberfl�che vertraut ist, �bernimmt der PC das Auswerten der Daten. Zus�tzlich werden die ausgewerteten Daten so gespeichert, dass sie schnell auffindbar sind und jederzeit erneut abgerufen werden k�nnen. 




\section{Ausgangslage - die erste Version des Tools}


to do: Ausz�ge aus Ausarbeitung von Simon und Co

TEST


\end{document}
